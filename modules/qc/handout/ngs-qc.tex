% Define the top matter
\setModuleTitle{Data Quality}
\setModuleAuthors{%
  Sonika Tyagi \mailto{sonika.tyagi@agrf.org.au}
}
\setModuleContributions{%
  Nathan S. Watson-Haigh \mailto{nathan.haigh@acpfg.com.au}
  Zhiliang Chen \mailto{zhiliang.chen@unsw.edu.au}
  Paula Moolhuijzen \mailto{paula.moolhuijzen@curtin.edu.au}
}

% Start: Module Title Page
\chapter{\moduleTitle}
\newpage
% End: Module Title Page

\section{Key Learning Outcomes}

After completing this practical the trainee should be able to:
\begin{itemize}
  \item Assess the overall quality of NGS (FastQ format) sequence reads
  \item Visualise the quality, and other associated matrices, of reads to decide
        on filters and cutoffs for cleaning up data ready for downstream analysis
  \item Clean up adaptors and pre-process the sequence data for further analysis
\end{itemize}

\section{Resources You'll be Using}
 
\subsection{Tools Used}
\begin{description}[style=multiline,labelindent=0cm,align=left,leftmargin=0.5cm]
  \item[FastQC]\hfill\\
  	\url{http://www.bioinformatics.babraham.ac.uk/projects/fastqc/}
  \item[Skewer]\hfill\\
   	\url{http://sourceforge.net/projects/skewer/}
  \item[FASTX-Toolkit]\hfill\\
 	\url{http://hannonlab.cshl.edu/fastx_toolkit/}
\end{description}

\section{Useful Links}
 
\begin{description}[style=multiline,labelindent=0cm,align=left,leftmargin=0.5cm]
  \item[FASTQ Encoding]\hfill\\
    \url{http://en.wikipedia.org/wiki/FASTQ_format#Encoding}
\end{description}

% \subsection{Sources of Data}
% TODO Provide a publically available data set used for this module
% \url{http://www.ebi.ac.uk/ena/data/view/ERR022484}\\
% \url{http://www.ebi.ac.uk/ena/data/view/ERR022485}

\newpage

\section{Introduction}

\begin{note}
Going on a blind date with your read set? For a better understanding of the
consequences please check the data quality!
\end{note}

For the purpose of this tutorial we are focusing only on Illumina sequencing
which uses 'sequence by synthesis' technology in a highly parallel fashion.
Although Illumina high throughput  sequencing provides highly accurate sequence
data, several sequence artifacts, including base calling errors and small
insertions/deletions, poor quality reads and primer/adapter contamination are
quite common in the high throughput sequencing data. The primary errors are
substitution errors. The error rates can vary from 0.5-2.0\% with errors mainly
rising in frequency at the 3' ends of reads.

One way to investigate sequence data quality is to visualize the quality scores
and other metrics in a compact manner to get an idea about the quality of a read
data set. Read data sets can be improved by pre processing in different ways
like trimming off low quality bases, cleaning up any sequencing adapters,
removing PCR duplicates and screening for contamination. We can also look at other statistics such
as, sequence length distribution, base composition, sequence complexity,
presence of ambiguous bases etc. to assess the overall quality of the data set.

Highly redundant coverage ($>$15X) of the genome can be used to correct sequencing
errors in the reads before assembly. Various k-mer based error
correction methods exist but are beyond the scope of this tutorial.

\subsection{Quality Value Encoding Schema}

In order to use a single character to encode Phred qualities, ASCII characters
are used (\url{http://shop.alterlinks.com/ascii-table/ascii-table-us.php}). All ASCII characters have a decimal
number associated with them but the first 32 characters are non-printable (e.g.
backspace, shift, return, escape). Therefore, the first printable ASCII   
character is number 33, the exclamation mark (!). In Phred+33 encoded quality
values the exclamation mark takes the Phred quality score of zero.

Early Solexa (now Illumina) sequencing needed to encode negative quality values.
Because ASCII characters $<$ 33 are non-printable, using the Phred+33 encoding was
not possible. Therefore, they simply moved the offset from 33 to 64 thus
inventing the Phred+64 encoded quality values. In this encoding a Phred quality
of zero is denoted by the ASCII number 64 (the @ character). Since Illumina 1.8,
quality values are now encoded using Phred+33.

FASTQ does not provide a way to describe what quality encoding is used for the
quality values. Therefore, you should find this out from your sequencing
provider. Alternatively, you may be able to figure this out by determining what
ASCII characters are present in the FASTQ file. E.g the presence of numbers in
the quality strings, can only mean the quality values are Phred+33 encoded.
However, due to the overlapping nature of the Phred+33 and Phred+64 encoding
schema it is not always possible to identify what encoding is in use. For
example, if the only characters seen in the quality string are (\texttt{@ABCDEFGHI}),
then it is impossible to know if you have really good Phred+33 encoded qualities
or really bad Phred+64 encoded qualities.

For a graphical representation of the different ASCII characters used in the two
encoding schema see: \url{http://en.wikipedia.org/wiki/FASTQ_format#Encoding}.

\section{Prepare the Environment}

\begin{information}
To investigate sequence data quality we will demonstrate tools called FastQC
and Skewer. FastQC will process and present the reports in a visual manner.
Based on the results, the sequence data can be processed using the Skewer.
We will use one data set in this practical, which can be found in the QC
directory on your desktop.
\end{information}

\begin{steps}

Open the Terminal and go to the directory where the data are stored:
\begin{lstlisting}
cd
ls
cd qc
pwd
\end{lstlisting}

At any time, help can be displayed for FastQC using the following command:
\begin{lstlisting}
fastqc -h
\end{lstlisting}

Look at SYNOPSIS (Usage) and options after typing fastqc -h 

\end{steps}

\section{Quality Visualisation}

\begin{information}
We have a file for a good quality and bad quality statistics. FastQC generates
results in the form of a zipped and unzipped directory for each input file.
\end{information}

\begin{steps}
Execute the following command on the two files:
\begin{lstlisting}
fastqc -f fastq qcdemo_R1.fastq.gz
fastqc -f fastq qcdemo_R2.fastq.gz
\end{lstlisting}

View the FastQC report file of the bad data using a web browser such as
firefox. The '\&' sign puts the job in the background.

\begin{lstlisting}
firefox qcdemo_R2_fastqc.html &
\end{lstlisting}

\end{steps}

\begin{note}
The report file will have a Basic Statistics table and various graphs and tables
for different quality statistics. E.g.:
\end{note}

\begin{table}[H]
  \centering
  \caption{FastQC Basic Statistics table}
    \begin{tabular}{ll}
    \toprule
    Filename & qcdemo\_R2.fastq.gz \\
    \midrule
    File type & Conventional base calls \\
    Encoding & Sanger / Illumina 1.9 \\
    Total Sequences & 1000000 \\
    Filtered Sequences & 0 \\
    Sequence length & 150 \\
    \%GC  & 37 \\
    \bottomrule
    \end{tabular}
  \label{tab:badexampleuntrimmed}
\end{table}

\begin{figure}[H]
\centering
\includegraphics[width=0.8\textwidth]{handout/bad_qcdemo_R2.png}
\caption{Per base sequence quality plot for \texttt{qcdemo\_R2.fastq.gz}.}
\label{fig:bad_example_untrimmed_plot}
\end{figure}

\begin{information}
A Phred quality score (or Q-score) expresses an error probability.  In particular, it
serves as a convenient and compact way to communicate very small error
probabilities.
The probability that base $A$ is wrong ($P(\sim A)$) is expressed
by a quality score, $Q(A)$, according to the relationship:
\\\\
$Q(A) =-10 log10(P(\sim A))$
\\\\
The relationship between the quality score and error probability is demonstrated
with the following table:

\begin{table}[H]
  \centering
  \caption{Error probabilities associated with various quality (Q) values}
    \begin{tabular}{rrr}
    \toprule
    \textbf{Quality score, Q(A)} & \textbf{Error probability, P($\sim$A)} & \textbf{Accuracy of the base call} \\
    \midrule
    10    & 0.1     & 90\% \\
    20    & 0.01    & 99\% \\
    30    & 0.001   & 99.9\% \\
    40    & 0.0001  & 99.99\% \\
    50    & 0.00001 & 99.999\% \\
    \bottomrule
    \end{tabular}
  \label{tab:quality_error_probs}
\end{table}

\end{information}

\begin{questions}
How many sequences were there in your file? What is the read length?
\begin{answer}
1,000,000. read length=150bp
\end{answer}

Does the quality score values vary throughout the read length?
(hint: look at the 'per base sequence quality plot')
\begin{answer}
Yes. Quality scores are dropping towards the end of the reads.
\end{answer}

What is the quality score range you see?
\begin{answer}
2-40
\end{answer}

At around which position do the scores start falling below Q20 for the 25\% quartile range (25\%of reads below Q20)? 
\begin{answer}
Around 30 bp position
\end{answer}

How can we trim the reads to filter out the low quality data?
\begin{answer}
By trimming off the bases after a fixed position of the read or by trimming off
bases based on the quality score.
\end{answer}
\end{questions}

\begin{bonus}
\subsection{Good Quality Data}
View the FastQC report files \texttt{fastqc\_report.html} to see examples of a good
quality data and compare the quality plot with that of the \texttt{bad\_example\_fastqc}.

\begin{lstlisting}
firefox qcdemo_R1_fastqc.html &
\end{lstlisting}
\end{bonus}

\begin{note}
Sequencing errors can complicate the downstream analysis, which normally
requires that reads be aligned to each other (for genome assembly) or to a
reference genome (for detection of mutations). Sequence reads containing errors
may lead to ambiguous paths in the assembly or improper gaps. In variant
analysis projects sequence reads are aligned against the reference genome. The
errors in the reads may lead to more mismatches than expected from
mutations alone. But if these errors can be removed or corrected, the read
alignments and hence the variant detection will improve. The assemblies will also
improve after pre-processing the reads to remove errors.
\end{note}

\section{Read Trimming}
Read trimming can be done in a variety of different ways. Choose a method
which best suits your data. Here we are giving examples of fixed-length trimming
and quality-based trimming.

\subsection{Quality Based Trimming}
Base call quality scores can be used to dynamically determine the trim
points for each read. A quality score threshold and minimum read length
following trimming can be used to remove low quality data.


\begin{steps}

The previous FastQC results show R1 is fine but R2 has low quality at the end. There is no adaptor contamination though.
We will be using Skewer to perform the quality trimming.

Run the following command to quality trim a set of paired end data.
\begin{lstlisting}
cd /home/trainee/qc
skewer -t 4 -l 50  -q 30 -Q 25 -m pe -o qcdemo qcdemo_R1.fastq.gz qcdemo_R2.fastq.gz
\end{lstlisting}
\end{steps}


\begin{note}
\begin{description}[style=multiline,labelindent=0cm,align=right,leftmargin=\descriptionlabelspace,rightmargin=1.5cm,font=\ttfamily]
\item[-t] number of threads to use
\item[-l] min length to keep after trimming
\item[-q] Quality threshold used for trimming at 3' end
\item[-Q] mean quality threshold for a read
\item[-m] pair-end mode
\end{description}
\end{note}

\begin{steps}
Run FastQC on the quality trimmed file and visualise the quality scores. 

Look at the last files generated, are the file names same as the input ?
\begin{lstlisting}
ls -ltr
\end{lstlisting}

Run Fastqc on the quality trimmed files:
\begin{lstlisting}
fastqc -f fastq qcdemo_R1.fastq-trimmed-pair1.fastq
fastqc -f fastq qcdemo_R1.fastq-trimmed-pair2.fastq
\end{lstlisting}

Visualise the fastqc results:
\begin{lstlisting}
firefox qcdemo_R1.fastq-trimmed-pair1_fastqc.html &
firefox qcdemo_R1.fastq-trimmed-pair2_fastqc.html&
\end{lstlisting}
Let's look at the quality from the second reads. The output should look like:

\begin{table}[H]
  \centering
  \caption{FastQC Basic Statistics table}
    \begin{tabular}{ll}
    \toprule
    Filename & qcdemo\_R1.fastq-trimmed-pair2.fastq\\
    \midrule
    File type & Conventional base calls\\
    Encoding & Sanger / Illumina 1.9\\
    Total Sequences & 742262\\
    Filtered Sequences & 0\\
    Sequence length & 50-150\\
    \%GC & 37\\
    \bottomrule
    \end{tabular}
  \label{tab:badexamplequalitytrimmed}
\end{table}

\begin{figure}[H]
\centering
\includegraphics[width=0.8\textwidth]{handout/bad_qcdemo_R2_quality_trimmed.png}
\caption{Per base sequence quality plot for the quality-trimmed \texttt{qcdemo\_R2.fastq.gz} reads.}
\label{fig:bad_example_quality_trimmed_plot}
\end{figure}

\end{steps}

\begin{questions}

Did the number of total reads in R1 and R2 change after trimming?
\begin{answer}
Quality trimming discarded $>$25000 reads. However, We retain a lot of maximal
length reads which have good quality all the way to the ends.
\end{answer}

What reads lengths were obtained after quality based trimming?
\begin{answer}
50-150

Reads $<$50 bp, following quality trimming, were discarded.
\end{answer}

Did you observe adapter sequences in the data?
\begin{answer}
No. (Hint: look at the overrepresented sequences.
\end{answer}

How can you use -a option with fastqc ? (Hint: try fastqc -h).
\begin{answer}
Adaptors can be supplied in a file for screening.
\end{answer}
\end{questions}

\begin{advanced}
\subsection{Adapter Clipping}
Sometimes sequence reads may end up getting the leftover of adapters and primers
used in the sequencing process. It's good practice to screen your data for
these possible contamination for more sensitive alignment and assembly based
analysis.

\begin{note}
This is particularly important when read lengths can be longer than the
molecules being sequenced. For example when sequencing miRNAs.
\end{note}

Various QC tools are available to screen and/or clip these adapter/primer
sequences from your data. Apart from skewer which will be using today the 
following two tools are also useful for trimming and removing adapter sequence.

Cutadapt 
\url{http://code.google.com/p/cutadapt/}

Trimmomatic 
\url{http://www.usadellab.org/cms/?page=trimmomatic}

\begin{steps}
Here we are demonstrating \texttt{Skewer} to trim a given adapter
sequence.

\begin{lstlisting}
cd /home/trainee/qc
fastqc -f fastq  adaptorQC.fastq.gz
firefox adaptorQC_fastqc.html 
skewer -x TGGAATTCTCGGGTGCCAAGGT -t 20 -l 10 -L 35 -q 30 adaptorQC.fastq.gz
\end{lstlisting}
\end{steps}

\begin{note}
\begin{description}[style=multiline,labelindent=0cm,align=right,leftmargin=\descriptionlabelspace,rightmargin=1.5cm,font=\ttfamily]
\item[-x] adaptor sequence used
\item[-t] number of threads to use
\item[-l] min length to keep after trimming
\item[-L] Max length to keep after trimming, in this experiment we were expecting only small RNA fragments
\item[-Q] Quality threshold used for trimming at 3' end. Use -m option to control the end you want to trim
\end{description}
\end{note}

\begin{steps}
Run FastQC on the adapter trimmed file and visualise the quality scores. Fastqc now shows adaptor free results.

\begin{lstlisting}
fastqc adaptorQC.fastq-trimmed.fastq
firefox adaptorQC.fastq-trimmed_fastqc.html &
\end{lstlisting}
\end{steps}

\subsection{Fixed Length Trimming}
\textbf{We will not cover Fixed Length Trimming but
provide the following for your information.}
Low quality read ends can be trimmed using a fixed-length trimming. We will use the
\texttt{fastx\_trimmer} from the FASTX-Toolkit. Usage message to find out various options
you can use with this tool. Type \texttt{fastx\_trimmer -h} at anytime to display help.

\begin{steps}
We will now do fixed-length trimming of the \texttt{bad\_example.fastq} file
using the following command. You should still be in the qc directory, if not cd back in.
\begin{lstlisting}
cd /home/trainee/qc
fastqc -f fastq bad_example.fastq 
fastx_trimmer -h
fastx_trimmer -Q 33 -f 1 -l 80 -i bad_example.fastq -o bad_example_trimmed01.fastq
\end{lstlisting}
\end{steps}

\begin{note}
We used the following options in the command above:
\begin{description}[style=multiline,labelindent=0cm,align=right,leftmargin=\descriptionlabelspace,rightmargin=1.5cm,font=\ttfamily]
 \item[-Q 33] Indicates the input quality scores are Phred+33 encoded
 \item[-f] First base to be retained in the output
 \item[-l] Last base to be retained in the output
 \item[-i] Input FASTQ file name
 \item[-o] Output file name
\end{description}
\end{note}

\begin{steps}
Run FastQC on the trimmed file and visualise the quality scores of the trimmed file.
\begin{lstlisting}
fastqc -f fastq bad_example_trimmed01.fastq
firefox bad_example_trimmed01_fastqc.html &
\end{lstlisting}

The output should look like:

\begin{table}[H]
  \centering
  \caption{FastQC Basic Statistics table}
    \begin{tabular}{ll}
    \toprule
    Filename & bad\_example\_trimmed01.fastq\\
    \midrule
     File type & Conventional base calls\\
     Encoding & Sanger / Illumina 1.9\\
     Total Sequences & 40000\\
     Filtered Sequences & 0\\
     Sequence length & 80\\
    \%GC & 48\\
    \bottomrule
    \end{tabular}
  \label{tab:badexampletrimmed}
\end{table}

\begin{figure}[H]
\centering
\includegraphics[width=0.8\textwidth]{handout/bad_example_trimmed_to_80bp.png}
\caption{Per base sequence quality plot for the fixed-length trimmed \texttt{bad\_example.fastq} reads.}
\label{fig:bad_example_trimmed_plot}
\end{figure}

\end{steps}

\begin{questions}
What values would you use for \texttt{-f} if you wanted to trim off 10 bases at
the 5' end of the reads?
\begin{answer}
\texttt{-f 11}
\end{answer}
\end{questions}
\end {advanced}
